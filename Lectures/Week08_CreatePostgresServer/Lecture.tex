\documentclass[12pt,a4paper]{article}
\usepackage{lipsum}
\usepackage{authblk}
\usepackage[marginparwidth=1.5in,top=2cm, bottom=2cm, left=2cm, right=2.5in,
marginparsep=1cm]{geometry}

\PassOptionsToPackage{hyphens}{url}\usepackage{hyperref}

\newcommand{\lectnote}[1]{\marginpar{\texttt{{\small{ #1 } }}}}

\usepackage{color}

\usepackage{multicol}
\usepackage{textcomp}
\usepackage{color}
\usepackage{graphicx}
\definecolor{pblue}{rgb}{0.13,0.13,1}
\definecolor{pgreen}{rgb}{0,0.5,0}
\definecolor{pred}{rgb}{0.9,0,0}
\definecolor{pgrey}{rgb}{0.46,0.45,0.48}

\usepackage{listings}
\lstset{language=SQL,
  showspaces=false,
  showtabs=false,
  breaklines=true,
  showstringspaces=false,
  breakatwhitespace=true,
  commentstyle=\color{pgreen},
  keywordstyle=\color{pblue},
  stringstyle=\color{pred},
  basicstyle=\ttfamily,
  frame=single,
  moredelim=[il][\textcolor{pgrey}]{$$},
  moredelim=[is][\textcolor{pgrey}]{\%\%}{\%\%}
}

\lstdefinestyle{term}{language=bash,
  columns=fullflexible,
  showspaces=false,
  showtabs=false,
  breaklines=true,
  showstringspaces=false,
  tabsize=2,
  breakatwhitespace=true,
  commentstyle=\color{pgreen},
  keywordstyle=\color{pblue},
  stringstyle=\color{pred},
  basicstyle=\small\ttfamily,
  frame=single,
  moredelim=[il][\textcolor{pgrey}]{$$},
  moredelim=[is][\textcolor{pgrey}]{\%\%}{\%\%},
  upquote=true
}
\lstdefinestyle{sh}{language=bash,
  columns=fullflexible,
  showspaces=false,
  showtabs=false,
  breaklines=true,
  showstringspaces=false,
  tabsize=2,
  breakatwhitespace=true,
  commentstyle=\color{pgreen},
  keywordstyle=\color{pblue},
  stringstyle=\color{pred},
  numbers=left,
  stepnumber=1,
  basicstyle=\small\ttfamily,
  frame=single,
  moredelim=[il][\textcolor{pgrey}]{$$},
  moredelim=[is][\textcolor{pgrey}]{\%\%}{\%\%},
  upquote=true
}


\usepackage[T1]{fontenc}

\newcommand{\schedule}[2]{\textbf{#1} \textit{#2}}

\usepackage{fancyhdr}
%
\pagestyle{fancy}
%
\renewenvironment{abstract}{%
\hfill\begin{minipage}{0.95\textwidth}
\rule{\textwidth}{1pt}}
{\par\noindent\rule{\textwidth}{1pt}\end{minipage}}
%
\makeatletter
\renewcommand\@maketitle{%
\hfill

\begin{minipage}{0.95\textwidth}
\vskip 1em
\let\footnote\thanks 
{\LARGE \@title \par }
\vskip 0.5em
%{\large \@author \par}
%\vskip 0.5em
{\large \@date \par}
\end{minipage}
\vskip 1em \par
}

\makeatother
%
\begin{document}
%
%title and author details
\title{\textbf{Week08 Create a PostgreSQL Server}}
%\author[1]{Melvyn Ian Drag}
\date{\today}
%
\maketitle

\begin{abstract}
In the last lecture we played around with a lightweight database; it's aptly
called `sqlite'. This week we'll configure a database server running PostgreSQL.
You'll have to use some of your Linux skills to get the server configured, and
then you'll see that this database is similar - though not identical - to
sqlite3. After successfully completing this week's homework you can slap
`PostgreSQL' next to `sqlite3' in the \textit{skills} section of your resume.
\end{abstract}

\section{Warning!}
As usual with computer programs, all of the commands in this document must be executed precisely as they are found herein. If an when you come across an error, please let me know so I can update the notes. All of the information in these notes was condensed from the following references:

\begin{enumerate}
\item \url{https://www.digitalocean.com/community/tutorials/how-to-install-and-use-postgresql-on-ubuntu-18-04}
\item \url{http://www.postgresqltutorial.com/postgresql-inner-join/}
\item \url{https://dba.stackexchange.com/questions/190674/can-a-database-table-not-have-a-primary-key}
\item \url{https://www.postgresql.org/message-id/001f01c018c2$830133b0$64898cd5@northlink.gr}
\item \url{https://support.plesk.com/hc/en-us/articles/115003321434-How-to-enable-remote-access-to-PostgreSQL-server-on-a-Plesk-server-}
\end{enumerate}

\section{What's a database?}
Need to fill this in. For now, just give a rough explanation.
Show the inner join table and doodle on the board what we expect.

By the way, Postgres is a Relational Database. You may have heard word like "Microsoft Access", "Oracle", "PostgreSQL", "SQLite3", "mySQL" and more. These are all relational databases. They all have slightly different implementations, and the way you interact with them is *slightly* different, but they all accomplish the same task. They store tabular data and allow you to manipulate the tables. The language you use to interact with these databases is called "SQL", though, as I said, the implementations of SQL in these different databases is *slightly* different in each implementation. So the stuff I'm going to show you today is conceptually portable between these different DBs, but you'll need to google the syntax. If you've used mysql before you may notice that the commands we type thorughout this lecture are a teensy bit different from mysql, but approximately the same.

\subsection{This is a Linux Class not a SQL class, I hate this lecture}
Linux is useful on laptops as you can see - we are all doing our regular computer stuff on linux laptops. Linux is also hugely popular as a server OS. Many people run windows on their personal laptops, for example, but still use Linux to run their websites, source control (e.g. git) servers, proxy servers, and - you guessed it - database servers. SQL databases ( aka relational databases ) are used in business, on website backends, and probably other places too. In your linux career you will likely be interacting with a database in some make, shape or form, so listen up! This is useful info, although you might not realize this for another two years. Thank me later.


To be more explicit when you sign up on a website for example it asks for you first name, last name, password, etc.. When you click 'submit' on the webpage, that info goes to some website code and then flows into a database. When you login again, somehow the website knows your name - it gets that info out of a database. 

\section{What is our goal today?}
We want to be able to do the inner join operation we just sketched out remotely, from our computer, on a database that lives on a different linux computer.

\section{Install PostgreSQL}

\subsection{Intro}
Follow instuctions here precisely:
\url{https://www.digitalocean.com/community/tutorials/how-to-install-and-use-postgresql-on-ubuntu-18-04}

Then work through the sample exercises.

\subsection{Create a new droplet}
Create a new debian 10 digital ocean droplet. Make sure you add your ssh key to the droplet as we go along.

\subsection{Install postgress}
\begin{lstlisting}[language=bash]
user@laptop$ ssh root@ipaddr
root@ipaddr$ apt update
# stuff happens
root@ipaddr$ apt install postgresql postgresql-contrib
# stuff happens. Say yes when prompted.
root@ipaddr$ sudo -i -u postgres
postgres@ipaddr$ createuser --interactive
# add user 'test'
# make user a superuser
postgres@ipaddr$ createdb test # make this the same name as the username you created
postgres@ipaddr$ exit
root@ipaddr$ adduser test # same name as db
root@ipaddr$ sudo -i -u  test
test@ipaddr$ psql # this will open a different looking ocmmand prompt. That is the postgres prompt
test=# 
test=# \conninfo
# youll see some info about your database connection
test=# \password
# enter a new password. Note this somewhere.
# exit by typing \q
# reenter by typing psql as before.
\end{lstlisting}

Your database is installed an functioning! Now we can move on.

\section{ Play With Your Database }

\subsection{Introduction}
Last week I gave you a prepopulated sqlite3 database that I found on the
internet. This week we're creating tables inside a new database. I \textit{think}
all you know about SQL at this point is \textbf{SELECT} along with some
conditions. Now we'll learn how to Create tables, add rows to tables and do a
join.


\subsection{Create Table store}
Create the table like this:
\begin{lstlisting}
CREATE TABLE store (
	store_id   int PRIMARY KEY,
	price      int NOT NULL,
	about_item varchar (25) NOT NULL
);
\end{lstlisting}

Add some stuff to the table like this:

\begin{lstlisting}
INSERT INTO store ( store_id, price, about_item ) VALUES ( 1, 100, 'Fancy dress shirt' );
INSERT INTO store ( store_id, price, about_item ) VALUES ( 2, 40, 'reasonable shirt');
INSERT INTO store ( store_id, price, about_item ) VALUES ( 3, 40, 'branded tshirt');
\end{lstlisting}

Note that if you try to add an element with a primary key that is already there it will yell at you:
\begin{lstlisting}
INSERT INTO store ( store_id, price, about_item ) VALUES ( 3, 20, 'another shirt');
\end{lstlisting}

Server says:
\begin{verbatim}
TODO. THE SERVER WILL GIVE AN ERROR HERE FOR THE REASON DESCRIBED BELOW
\end{verbatim}

That's because the PRIMARY KEY column - in this case store\_id - should uniquely
identify a product. Verify the stuff in your table using the SELECT command that
we learned last week.

\subsection{SELECT}
Let's look in the store table.

\begin{lstlisting}
SELECT * FROM store;
\end{lstlisting}

See? This command above is like the one in sqlite. You can also try the other
things you know:

\begin{itemize}
\item ORDER BY ( ASC/DESC )
\item LIMIT
\item WHERE
\item OR
\item AND
\end{itemize}

As I'm writing these notes I'm not sure exactly how these ideas are written in
PostgreSQL. Keep in mind these operations are supported across different
databases ( e.g. Access, mySQL, Postgres, sqlite3, etc. ) but the SQL syn tax
may vary slightly across the different databases.

\subsection{ Create Table warehouse }

\begin{lstlisting}
CREATE TABLE warehouse(
	uniq_id      int      PRIMARY KEY,
	quantity     smallint NOT NULL,
	warehouse_id int NOT NULL,
	FOREIGN KEY (warehouse_id) REFERENCES store (store_id)
);
\end{lstlisting}

And then we'll put some stuff in the table

\begin{lstlisting}
INSERT INTO warehouse ( uniq_id, quantity, warehouse_id ) VALUES ( 1, 10, 1 );
INSERT INTO warehouse ( uniq_id, quantity, warehouse_id ) VALUES ( 2, 30, 2 );
INSERT INTO warehouse ( uniq_id, quantity, warehouse_id ) VALUES ( 3, 15, 3 );
SELECT * FROM warehouse;
\end{lstlisting}

\subsection{BTW}
By the way, these sql databases are called \textbf{Relational Databases}. The
idea behind them is, as we saw last week, to store data in tables kind of like
excel spreadsheets. We say relational, because the tables are related! Look at
the schema image I gave you last week for the murder mystery db. There are
``keys'' and lines connecting certain columns in the database. These describe
how tables are related. 

\subsection{ Your First Inner Join }

There are a few types of joins in SQL. One is an inner join. This is when you
join two tables together based on values in a table

\begin{lstlisting}
SELECT 
	store.store_id,
	warehouse.warehouse_id,
	store.price,
	store.about_item,
	warehouse.quantity
FROM
	store
INNER JOIN
	warehouse
ON
	store.store_id = warehouse.warehouse_id;
\end{lstlisting}

Output should look like this:

\begin{verbatim}
 store_id | warehouse_id | price |    about_item     | quantity 
----------+--------------+-------+-------------------+----------
        1 |            1 |   100 | Fancy dress shirt |       10
        2 |            2 |    40 | reasonable shirt  |       30
        3 |            3 |    40 | branded tshirt    |       15
\end{verbatim}

\section{Case sensitivity}
Is the SQL dialect used in Postgres case insensitive? We can check and find out.
I think it is, but just to be sure we can look on google and then do a few
experiments to make sure.

\section{ Configure server to be accessible on the internet }
At this point, we have a database on our computer with some interesting tables in it. We've been able to add and manipulate data in the database. Now we just want to configure our server such that we can access the database remotely from the terminal on another computer.

Right now you cannot access this database. By default, postgreSQL only makes this database visible from the machine running the database server. Now we are going to expose it to the internet. You need to modify two files:

REad this stuff in case there is a mistake in lecture notes
\begin{enumerate}
\item \url{https://bosnadev.com/2015/12/15/allow-remote-connections-postgresql-database-server/}
\item \url{https://support.plesk.com/hc/en-us/articles/115003321434-How-to-enable-remote-access-to-PostgreSQL-server-on-a-Plesk-server-}
\end{enumerate}

\subsection{postgresql.conf}
Find this file on your computer by typing:

\begin{lstlisting}[language=bash]
find / -name postgresql.conf
\end{lstlisting}

For me the file is here. Yours should be in the same spot.

\begin{verbatim}
/etc/postgresql/11/main/postgresql.conf 
\end{verbatim}

Find the line in the file that says 'listen\_addresses' and change it such that it says:

\begin{verbatim}
listen_addresses = '*'
\end{verbatim}

Note that postgres runs on port 5432.

\subsection{pg\_hba.conf}
 For me the file is here:

\begin{verbatim}
/etc/postgresql/11/main/pg_hba.conf 
\end{verbatim}

Since we're all on a digital ocean debian 10 server, your location should be the same, but maybe not. 

Add the following line to the bottom of that file:

\begin{verbatim}
host    all     all  0.0.0.0/0    md5
\end{verbatim}

\subsection{Restart postgres}
Run
\begin{lstlisting}
root@machine$ service postgresql restart
\end{lstlisting}

\subsection{Connect remotely}

Back on your laptop install a postgres client that can talk to the database. On ubuntu you install this with:

\begin{verbatim}
user@laptop$ sudo apt install postgresql-client-common postgresql-client
\end{verbatim}

Gather your server ip address, database user name, database user password, and run the following command:

\begin{verbatim}
psql -h 67.205.173.100 -p 5432 -U test
\end{verbatim}

But supply your server's ip address and postgres username, if it's different
from test.

Enter your password when prompted.

Then run this command at the prompt:

\begin{verbatim}
user# \dt
\end{verbatim}

You should see the tables you created. You can now rerun your inner join command, select commands, whatever database things you want to do, remotely.

\section{Column datatypes}
There is so much to know about databases. When we created tables I made the column types int, varchar, foreign key, primary key, small int. You can read more about this here:
\url{http://www.postgresqltutorial.com/postgresql-data-types/}

Let's create tables with some other datatypes in it. ( 10 minute exercise or so )

\section{ What's a Primary Key and Do I need One?}
Yes, all tables should have a primary key. Want more information? Read this
link.
\url{https://dba.stackexchange.com/questions/190674/can-a-database-table-not-have-a-primary-key}

\section{ Questions }
You should feel confused a bit. At this point there are more questions than
answers - there is much much more to know about SQL. I think  this bit of
knowledge is enough to get you started - now go forth and look  for more
answers! There are tons of great youtube references and books about sqlite3 and
PostgreSQL that will fill in many of the cracks left by this lecture.

\section{One More Thing}

Jackie, one more thing ~\ref{fig:jackie}.

\begin{figure}[h]
\centering
	\includegraphics[width=0.8\textwidth]{Images/jackie.png}
	\caption{{\small \texttt{One more thing! We'll use curl next week. I don't
want next week to be the first time you've heard of it.}}}
	\label{fig:jackie}
\end{figure}

\subsection{curl}
We'll talk more about curl at a later date. It's a powerful tool you use for
sending data over the internet. You can use curl to check your ipaddress like
this:

\begin{lstlisting}[language=bash]
user@machine$ curl ipinfo.io/ip
a.bb.c.ddd
\end{lstlisting}

The ipaddress should be the same as the one you used when you connected to the
server with ssh.
\end{document}
