\documentclass[12pt,a4paper]{article}
\usepackage{lipsum}
\usepackage{epigraph}
\usepackage{authblk}
\usepackage[marginparwidth=1.5in,top=2cm, bottom=2cm, left=2cm, right=2.0in,
marginparsep=1cm]{geometry}

\PassOptionsToPackage{hyphens}{url}
\usepackage{hyperref}

\usepackage{multicol}
\usepackage{textcomp}
\usepackage{color}
\usepackage{graphicx}
\definecolor{pblue}{rgb}{0.13,0.13,1}
\definecolor{pgreen}{rgb}{0,0.5,0}
\definecolor{pred}{rgb}{0.9,0,0}
\definecolor{pgrey}{rgb}{0.46,0.45,0.48}

\newcommand{\lectnote}[1]{\marginpar{\texttt{{\small{ #1 } }}}}

\usepackage{listings}
\lstdefinestyle{java}{
  language=Java,
  columns=fullflexible,
  showspaces=false,
  showtabs=false,
  breaklines=true,
  showstringspaces=false,
  tabsize=2,
  breakatwhitespace=true,
  commentstyle=\color{pgreen},
  keywordstyle=\color{pblue},
  stringstyle=\color{pred},
  basicstyle=\small\ttfamily,
  numbers=left,
  stepnumber=1,
  frame=shadowbox,
  moredelim=[il][\textcolor{pgrey}]{$$},
  moredelim=[is][\textcolor{pgrey}]{\%\%}{\%\%},
  upquote=true
}

\lstdefinestyle{term}{language=bash,
  columns=fullflexible,
  showspaces=false,
  showtabs=false,
  breaklines=true,
  showstringspaces=false,
  tabsize=2,
  breakatwhitespace=true,
  commentstyle=\color{pgreen},
  keywordstyle=\color{pblue},
  stringstyle=\color{pred},
  basicstyle=\small\ttfamily,
  frame=single,
  moredelim=[il][\textcolor{pgrey}]{$$},
  moredelim=[is][\textcolor{pgrey}]{\%\%}{\%\%},
  upquote=true
}
\lstdefinestyle{sh}{language=bash,
  columns=fullflexible,
  showspaces=false,
  showtabs=false,
  breaklines=true,
  showstringspaces=false,
  tabsize=2,
  breakatwhitespace=true,
  commentstyle=\color{pgreen},
  keywordstyle=\color{pblue},
  stringstyle=\color{pred},
  numbers=left,
  stepnumber=1,
  basicstyle=\small\ttfamily,
  frame=single,
  moredelim=[il][\textcolor{pgrey}]{$$},
  moredelim=[is][\textcolor{pgrey}]{\%\%}{\%\%},
  upquote=true
}

\lstdefinestyle{py}{language=python,
  columns=fullflexible,
  showspaces=false,
  showtabs=false,
  breaklines=true,
  showstringspaces=false,
  tabsize=2,
  breakatwhitespace=true,
  commentstyle=\color{pgreen},
  keywordstyle=\color{pblue},
  stringstyle=\color{pred},
  numbers=left,
  stepnumber=1,
  basicstyle=\small\ttfamily,
  frame=single,
  moredelim=[il][\textcolor{pgrey}]{$$},
  moredelim=[is][\textcolor{pgrey}]{\%\%}{\%\%},
  upquote=true
}

\lstdefinestyle{txt}{
  columns=fullflexible,
  showspaces=false,
  showtabs=false,
  breaklines=true,
  showstringspaces=false,
  tabsize=2,
  breakatwhitespace=true,
  numbers=left,
  stepnumber=1,
  basicstyle=\small\ttfamily,
  frame=single,
  moredelim=[il][\textcolor{pgrey}]{$$},
  moredelim=[is][\textcolor{pgrey}]{\%\%}{\%\%},
  upquote=true
}

\usepackage[T1]{fontenc}

\newcommand{\schedule}[2]{\textbf{#1} \textit{#2}}

\usepackage{fancyhdr}
%
\pagestyle{fancy}
%
\renewenvironment{abstract}{%
\hfill\begin{minipage}{0.95\textwidth}
\rule{\textwidth}{1pt}}
{\par\noindent\rule{\textwidth}{1pt}\end{minipage}}
%
\makeatletter
\renewcommand\@maketitle{%
\hfill

\begin{minipage}{0.95\textwidth}
\vskip 1em
\let\footnote\thanks 
{\LARGE \@title \par }
\vskip 0.5em
%{\large \@author \par}
%\vskip 0.5em
{\large \@date \par}
\end{minipage}
\vskip 1em \par
}

\makeatother

\title{\textbf{Week 03} \\
ssh, grep, regexes, if, and bash scripts
}
\author{
	Melvyn Ian Drag
}
\date{\today}




\begin{document}
\maketitle

\begin{abstract}
Tonight we'll look at grep a bit more, we'll learn what a regex is, youll see
how if statements work in bash, we'll see a tidy way to write bash commands, and
you'll use a cool tool called ssh.
\end{abstract}

\section{Review of the commands}
By this point we've learned a whole bunch of basic command line programs you
can use on Linux. Spend a few minutes reviewing them in class.

\section{Vim Recap}
There is alot to know about Vim, it's a great editor.

The main thing to know is

\begin{lstlisting}[style=term]
root@machine$ vim filename # this opens a file called "filename"
\end{lstlisting}

When you are in vim, you need to know two things to get started:


\begin{itemize}
\item When you first launch vim you are in NORMAL mode. You cannot type into
your file yet.
\item Hit "i" on the keyboard to enter INSERT mode. In INSERT mode you can type
in your file.
\item When you are done working on your file, hit ESC. This puts you back in
NORMAL mode
\item In NORMAL mode you can save and quit, save, or just quit.
\item To save and quit type ":wq".
\item To quit without saving type ":q!"
\item To save without quitting type ":w". You are still in NORMAL mode. Remember
if you want to go back to editing the file, type "i". This will put you back in
INSERT mode.
\end{itemize}


\section{Review SSH}

We won't go too in depth. If you want to go deep, \href{
https://www.tiltedwindmillpress.com/product/ssh-mastery-2nd-edition/}{check out
Open SSH Mastery, an awesome book by Michael Lukas}.

Here's what I'll tell you though:

\begin{itemize}
\item Open a terminal and type "ssh-keygen" to generate a key pair.
\item The keypair will be stored in the .ssh folder in your home directory.
\item Guard the id\_rsa with your life. If anyone gets that file, they can hack
your servers.
\item If you want to connect to a server with ssh ( without using a password ),
copy the contents of the id\_rsa.pub file to a line in the .ssh/authorized\_keys
file.
\item With respect to the last point - I typically put 3 keys on my servers. One
for my mac, one for my windows pc, and one for my linux laptop. That way I can
connect to my server from any pc in my house. If you use multiple computers, you
might want to generate a key pair on each one, and put all  the public keys on
your server.
\item To connect with ssh, do this: \textit{ssh username\@machineIPaddress}.
\item So far we are always connecting as a user named "root". This will change
in the coming weeks.
\end{itemize}

\noindent\fbox{%
	\parbox{\textwidth}{%
		\textbf{Class Example}\newline
		Have everyone in class send me their pub keys. Put the pub keys on my
server. Show they can all connect. This might be chaos with too  many students.
In the case of many students, just do this with a handful
}%
}


\section{How To Download Files off the Internet}

Two good choices: \textbf{wget} and \textbf{curl}.

\subsection{wget}
If you want to download this file:\newline

\url{https://www.cc.gatech.edu/~spencer/poems/woods.txt}\newline

Then you use this command:
\begin{lstlisting}[style=term]
root@machine$ wget https://www.cc.gatech.edu/~spencer/poems/woods.txt
root@machine$ ls
# you will see a file called woods.txt
\end{lstlisting}

If you would like to rename the file from the name it has on the webserver you
downloaded from, use the \textbf{O} flag.


\begin{lstlisting}[style=term]
root@machine$ wget -O robertFrost.poem  https://www.cc.gatech.edu/~spencer/poems/woods.txt
root@machine$ ls
# you will see a file called robertFrost.poem
\end{lstlisting}

\noindent\fbox{%
	\parbox{\textwidth}{%
		\textbf{Class Example}\newline
		Go to the url above in a browser and show that there is a text file on
the internet. Run wget and show it downloads to the computer.
	}%
}


\subsection{cURL}

You can also use a tool called curl. cURL is a very powerful tool that we will
learn in depth later in the semester. I'll just tell you that it exists now to
plant a seed in your memory.

Here's how you download a file with curl:

\begin{lstlisting}[style=term]
root@machine$ curl https://www.cc.gatech.edu/~spencer/poems/woods.txt --output
frost.poem
root@machine$ ls
frost.poem
\end{lstlisting}

Now that's enough about downloading files, let's talk about grep.

\section{grep}
Is for searching for strings in files. Do you want to know if the string "woods"
is in a file or set of files? \textbf{Use grep}. Do you want to know how many
times a particular number appears in a file? \textbf{Use grep}. Do you want to
do something more crazy like find if the word

\begin{verbatim}
goal
OR
gooaal
OR
gooooaaaal
OR
goooooooooooooooaaaaaaaaaaaaaaaaaal
\end{verbatim}

( or any other drawn out spelling ) exists in a file? \textbf{Use grep}.

Grep is commonly used with pipes. Like:

\begin{lstlisting}
root@machine$ history | grep wget
# this will list all the wget commands you ran.
\end{lstlisting}

It can do alot more than this simple trick though.

\subsection{Commonly used grep flags}
\begin{itemize}
\item \textbf{-i} Means "case insensitive". We can match the search pattern in
upper case or lowercase.
\item \textbf{-w} Means match only the whole word.
\item \textbf{-v} Inverse grep. Instead of matching, search for lines that DONT
match.
\item \textbf{-o} Usually grep returns the whole matching line. this returns
just the matching THING from the line.
\item \textbf{-n} Display not only the matching line, but also the line number.
\item \textbf{-r} Recursive. Search in subdirectories.
\item \textbf{-A} Include lines AFTER the matching line in addition to the
matching line.
\item \textbf{-B} Include lines BEFORE the matching line in addition to the
matching line.
\item \textbf{-C} Include lines both BEFORE and AFTER the matching line.
\end{itemize}


\begin{lstlisting}[style=term]
root@machine$ cat example_data.txt
hello world of grep
my favorite numbers are 1 2 3 4 5
HelLo WoRld of Greppp
root@machine$ grep "hello" example_data.txt
hello world of grep
root@machine$ grep -i "hello" example_data.txt
hello world of grep
HelLo WoRld of Greppp
root@machine$ # the next command wont match the last line because 
root@machine$ # the word grep != Greppp 
root@machine$ grep -iw "grep" example_data.txt 
hello world of grep
root@machine$ grep -v "of" example_data.txt
my favorite numbers are 1 2 3 4 5
root@machine$ # the next line only matches the 2, not the whole line
root@machine$ grep -o "2" example_data.txt
2
root@machine$ grep -nw hello example_data.txt
1:hello
\end{lstlisting}

The \textbf{r},\textbf{A},\textbf{B}, and \textbf{C} flags are important, lets
make sure too look back at them another time.

\subsection{Regular expressions and grep}
\epigraph{Some people, when confronted with a problem, think "I know, I'll use
regular expressions". Now they have two problems.}{Unknown source. // Programming
wisdom and humor.}

Grep stands for

\textbf{G}lobal \textbf{R}egular \textbf{E}xpression \textbf{P}rint.

Keep an open mind as we look at regular expressions in grep. I don't have all of
this committed to memory - I just know generally what they are for and know what
I need to look up when I'm stuck. 

Let's consider this file:

\begin{lstlisting}[style=term]
root@machine$ cat demo_file.txt
1
11
123
18888.111
hlo
hello
helllo
he1llo
\end{lstlisting}

Let's look at regular expressions to see if we can extract certain bits of
information.

\begin{lstlisting}[style=term]
$ grep "[0-9]" demo_file.txt # match any line with a digit from 0-9
1
11
123
18888.111
he1llo
$ grep "[0-9]\+" demo_file.txt # match any line with a series of digits
1
11
123
18888.111
he1lllo
$ grep -n "[0-9]\+" demo_file.txt # match just the series of digits (-o flag)
1
11
123
18888
111
1
$#this next one is tricky. 
$#match "h"
$#followed by an optional e ( e\? )
$#followed by any number of 1s and ls ([l1]\+)
$#followed by an obligatory o
$grep "he\?[l1]\+o" demo_file.txt 
hlo
hello
helllo
he1llo
$grep "he...o" demo_file.txt # find he followed by 3 anythings followed by o
helllo
he1llo
$ grep "^h" demo_file.txt # any line starting with h
hlo
hello
helllo
he1llo
$ grep "[0-9]$" demo_file.txt # any line ending in a digit.
1
11
123
18888.111
$ grep "1\{2,3\}" demo_file.txt # any line with between 2 and 3 ones.
11
123
18888.111
$ # this next one is tricky because it uses the caret with a different meaning.
$ grep "[^0-9]" demo_file.txt # any line with no digit in it
hlo
hello
helllo
\end{lstlisting}


These are the important characters you need to know:
\begin{itemize}
\item `.'
\item `\textbackslash?'
\item `\textbackslash+'
\item `\textbackslash\{'
\item `\textbackslash\}'
\item `\textbackslash|'
\end{itemize}

What to know:
\begin{enumerate}
\item The period (.) matches any single character.
\item \textbackslash? means that the preceding item is optional, and if found, will be matched at the most, once.
\item * means that the preceding item will be matched zero or more times.
\item \textbackslash+ means the preceding item will be matched one or more times.
\item {n} means the preceding item is matched exactly n times, while {n,} means the item is matched n or more times. {n,m} means that the preceding item is matched at least n times, but not more than m times. {,m} means that the preceding item is matched, at the most, m times.
\end{enumerate}

Some more syntax:
\begin{enumerate}
\item \textasciicircum (Caret)   =   match expression at the start of a line, as
in \textasciicircum\,A will match an A at the beginning of a line.
\item $ (Dollar sign)    =   match expression at the end of a line, as in A$.
\item \textbackslash (Back Slash)  =   turn off te special meaning of the next character, as in \^.
\item [ ] (Brackets)  =   match any one of the enclosed characters, as in [aeiou]. Use Hyphen "-" for a range, as in [0-9].
\item . (Period)  =   match a single character of any value, except end of line.
\item * (Asterisk)    =   match zero or more of the preceding character or expression.
\item \{x,y\} =   match x to y occurrences of the preceding.
\item \{x\}   =   match exactly x occurrences of the preceding.
\item  \{x,\}  =   match x or more occurrences of the preceding.
\item \begin{verbatim}[\^ ]    =   match any one character except those enclosed in [ ], as in
[\^ 0-9].\end{verbatim}
\end{enumerate}

That's it! So, given the file a.txt ( see this directory )

We can do the following

\begin{lstlisting}[style=term]
$ grep "a" a.txt # Find lines with an a
$ grep "a\?" a.txt # find lines with an optional a.
$ grep "a?" a.txt # find lines containing a?
$ grep "a\+" a.txt # find lines with 1 or more "a"s
$ grep "a+" a.txt # find lines  containing "a+" 
$ grep "a$" a.txt # find lines that end with a
$ grep "[0-9]$" a.txt # find lines that end with a number
$ grep "^[a-zA-Z]$" a.txt # find lines with one letter.
$ grep "a\{2,\}" a.txt # find lines with 2 or more "a"s
\end{lstlisting}

Regular expressions are usually called \textbf{regexes} ( rej - exes ). They are
useful for searching through text for patterns. Regular expressions are so
useful they are in
\hyperref{https://www.w3schools.com/python/python_regex.asp}{Python} and
\hyperref{https://docs.oracle.com/javase/7/docs/api/java/util/regex/Pattern.html}{Java}
and
\hyperref{https://developer.mozilla.org/en-US/docs/Web/JavaScript/Guide/Regular_Expressions}{javascript}
and virtually any other language you might become interested in.

In Linux land you will use regular expressions with tools like:
\begin{enumerate}
\item awk
\item sed
\item grep
\item perl
\end{enumerate}

Regular expressions are a fundamental part of what grep is for.  

\subsection{grep for things that aren't English or Numbers}

English isn't the only written language in the world. 

What if you are analyzing a piece of text in Brazilian Portuguese, and you want
to grep for the letter "ç"? How will  you grep for that with your English
language keyboard? The only way is to take a deep dive into Text Encodings.
You'll need to learn what the hexadecimal or binary representation of the letter
is, and you can use grep like that.

This also works for emojis! You can grep for emojis!

Later in the semester we will look into text encodings and how they affect the
various Linux text-processing tools.

We'll revisit this later.


\section{Bash Scripting}

We should close tonight by recognizing that bash is a programming language and
that you can write long scripts with bash. Up until this point we've only been
using bash interactively on the command line. Let's take a moment to look at 

YOU CAN IGNORE The NEXT SECTIONS for NOW

Just know the following:
\begin{enumerate}
\item Bash is a programming language
\item Bash has loops
\item Bash has if statements
\item Bash has arrays
\item Bash has variables
\end{enumerate}

We will visit this at a later date when it's more important.


Just do this:
\begin{lstlisting}[style=term]
$cat myfirstscript.sh
#!/bin/bash

mkdir -p a/b/c
echo "this is file a in diretory a" > a/a.txt
echo "this is file b in directory a/b" > a/b/b.txt
echo "this is file c in directory a/b/c" > a/b/c/c.txt
$ chmod +x myfirstscript.sh
$ ./myfirstscript.sh
$ls a
a.txt
$ls a/b
b.txt
$ls a/b/c
c.txt
\end{lstlisting}

Ignore the following sections if you are in Spring 2021 lecture.

\subsection{Introduction to if in bash}
Lots of programming languages have `ifs'. Java has them, python, C, C++,
javascript, and beyond. So does bash. It is a standard thing to do in
programming to check if a variable equals something.

The bash syntax for if is 

\begin{verbatim}
if CONDITION
then
 command
fi
\end{verbatim}

The tricky bit is getting the CONDITION part right, because it is different in
different situations, with different data types. I'll show you a little bit
about it right now. And, as always, I'll encourage you to go read more if you
want the full story. A good link is somewhere down below.

One more useful piece of information is that bash generally interprets values as strings, unless they can be used as numbers, in which case it assumes they are numbers. https://www.tldp.org/LDP/abs/html/untyped.html.

Comparison operators for numbers in bash are:

\begin{itemize}
\item -eq
\item -ne
\item -gt
\item -ge 
\item etc.
\end{itemize}

Comparison operators for strings are:

\begin{itemize}
\item =
\item !=
\item etc.
\end{itemize}

for more information see here
\url{https://www.tldp.org/LDP/abs/html/comparison-ops.html}


\begin{lstlisting}[style=term]
melvyn@thinkpad$ cat myFirstScript.sh
#!/bin/bash


#x1 and x2 are integers, though they could also be interpreted as strings.
x1=1
x2=2
if [ $x1 -lt $x2 ]        
then
    echo "$x1 < $x2"
else
    echo "$x2 <= $x1"
fi
melvyn@thinkpad$ bash myFirstScript.sh
1 < 2
\end{lstlisting}


Then this script will fail, because you are using an arithmetic comp on strings:

\begin{lstlisting}[style=term]
melvyn@laptop$ cat myBadScript.sh
x1=1a
x2=2a
if [ $x1 -lt $x2 ]        
then
    echo "$x1 < $x2"
else
    echo "$x2 <= $x1"
fi
melvyn@laptop$ bash myBadScript.sh
# error
\end{lstlisting}

You can easily fix this with:

\lstinputlisting[style=term, caption={option 1}]{option1.sh}

OR

\lstinputlisting[style=term, caption={option 2 for if with strings}]{option2.sh}


\textbf{Tip:} The old bash advice is to double quote all variables in bash to make sure they are interpreted as a single value.

\section{Fluff}
In case extra time, I solved a little problem today using tools you know.

I'm programming a little AVR microcontroller, and the tool you use to program it
is avrdude. If you type 
\begin{lstlisting}[style=term]
melvyn@laptop$ avrdude -c ?
\end{lstlisting}

it will list all the available hardware programmers out there. I wanted ot see
if my mkii was supported by this tool so I ran

\begin{lstlisting}[style=term]
melvyn@laptop$ avrdude -c ? | grep mkii
\end{lstlisting}

No results. Well I didn't know if avrdude uses upper or lower case, so I just
said like this:

\begin{lstlisting}[style=term]
melvyn@laptop$ avrdude -c ? | grep -i mkii
\end{lstlisting}

Still no results. Weird. But then I thought, how many types are there?

\begin{lstlisting}[style=term]
melvyn@laptop$ avrdude -c ? | wc -l
\end{lstlisting}

No results. But the command definitely outputs results. Do you know how to fix
the above commands?

ANSWER: avrdude -c ?  writes to stderr, not stdout ( for some reason ). So I
just route my stderr to stdout and then use the piped commands ( pipes work on
stdout, not stderr:

\begin{lstlisting}[style=term]
melvyn@laptop$ avrdude -c ? 2>&1 | grep -i mkii
# It comes out!
\end{lstlisting}

\end{document}




