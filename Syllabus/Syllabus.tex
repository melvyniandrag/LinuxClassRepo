% Don't touch this %%%%%%%%%%%%%%%%%%%%%%%%%%%%%%%%%%%%%%%%%%%
\documentclass[11pt]{article}
\usepackage{fullpage}
\usepackage[left=1in,top=1in,right=1in,bottom=1in,headheight=3ex,headsep=3ex]{geometry}
\usepackage{graphicx}
\usepackage{float}

\newcommand{\blankline}{\quad\pagebreak[2]}
%%%%%%%%%%%%%%%%%%%%%%%%%%%%%%%%%%%%%%%%%%%%%%%%%%%%%%%%%%%%%%

% Modify Course title, instructor name, semester here %%%%%%%%

\title{Linux Shell Programming}
\author{Melvyn Drag}
\date{Spring 2019}

%%%%%%%%%%%%%%%%%%%%%%%%%%%%%%%%%%%%%%%%%%%%%%%%%%%%%%%%%%%%%%

% Don't touch this %%%%%%%%%%%%%%%%%%%%%%%%%%%%%%%%%%%%%%%%%%%
\usepackage[sc]{mathpazo}
\linespread{1.05} % Palatino needs more leading (space between lines)
\usepackage[T1]{fontenc}
\usepackage[mmddyyyy]{datetime}% http://ctan.org/pkg/datetime
\usepackage{advdate}% http://ctan.org/pkg/advdate
\newdateformat{syldate}{\twodigit{\THEMONTH}/\twodigit{\THEDAY}}
\newsavebox{\MONDAY}\savebox{\MONDAY}{Mon}% Mon
\newcommand{\week}[1]{%
%  \cleardate{mydate}% Clear date
% \newdate{mydate}{\the\day}{\the\month}{\the\year}% Store date
  \paragraph*{\kern-2ex\quad #1, \syldate{\today} - \AdvanceDate[4]\syldate{\today}:}% Set heading  \quad #1
%  \setbox1=\hbox{\shortdayofweekname{\getdateday{mydate}}{\getdatemonth{mydate}}{\getdateyear{mydate}}}%
  \ifdim\wd1=\wd\MONDAY
    \AdvanceDate[7]
  \else
    \AdvanceDate[7]
  \fi%
}
\usepackage{setspace}
\usepackage{multicol}
%\usepackage{indentfirst}
\usepackage{fancyhdr,lastpage}
\usepackage{url}
\pagestyle{fancy}
\usepackage{hyperref}
\usepackage{lastpage}
\usepackage{amsmath}
\usepackage{layout}
\usepackage{multicol}


\lhead{}
\chead{}
%%%%%%%%%%%%%%%%%%%%%%%%%%%%%%%%%%%%%%%%%%%%%%%%%%%%%%%%%%%%%%

% Modify header here %%%%%%%%%%%%%%%%%%%%%%%%%%%%%%%%%%%%%%%%%
\rhead{\footnotesize Linux Shell Programming}

%%%%%%%%%%%%%%%%%%%%%%%%%%%%%%%%%%%%%%%%%%%%%%%%%%%%%%%%%%%%%%
% Don't touch this %%%%%%%%%%%%%%%%%%%%%%%%%%%%%%%%%%%%%%%%%%%
\lfoot{}
\cfoot{\small \thepage/\pageref*{LastPage}}
\rfoot{}

\usepackage{array, xcolor}
\usepackage{color,hyperref}
\definecolor{clemsonorange}{HTML}{EA6A20}
\hypersetup{colorlinks,breaklinks,linkcolor=clemsonorange,urlcolor=clemsonorange,anchorcolor=clemsonorange,citecolor=black}

\begin{document}

\maketitle

\blankline

\begin{tabular*}{.93\textwidth}{@{\extracolsep{\fill}}lr}

%%%%%%%%%%%%%%%%%%%%%%%%%%%%%%%%%%%%%%%%%%%%%%%%%%%%%%%%%%%%%%

% Modify information %%%%%%%%%%%%%%%%%%%%%%%%%%%%%%%%%%%%%%%%%
E-mail: \texttt{mdrag1@njcu.edu} & Web: \href{melvyniandrag.github.io}{\tt\bf melvyniandrag.github.io}  \\

 Office Hours: Th 6-7 \& Fri 10-1  &  Class Hours: Th 7-9:45pm \\

 Office: CS Adjunct Office & Class Room: TBD \\
 & \\
\hline
\end{tabular*}

\vspace{5 mm}

% First Section %%%%%%%%%%%%%%%%%%%%%%%%%%%%%%%%%%%%%%%%%%%%

\section*{Course Description}

In this fun, fast-paced class you're going to become a "Linux Rockstar". Within a few short months you'll get lots of great information to make you more employable and more efficient on Linux machines. We will learn about *nix operating systems and programming in this environment.  When possible we will explore topics in depth. More complex topics will be briefly touched upon to acquaint you with important *nix vocabulary words and key concepts.

% Second Section %%%%%%%%%%%%%%%%%%%%%%%%%%%%%%%%%%%%%%%%%%%

\section*{Required Materials}

\begin{itemize}
\item Access to a Google Cloud Virtual Machine ( provided and free ).
\end{itemize}

% Fourth Section %%%%%%%%%%%%%%%%%%%%%%%%%%%%%%%%%%%%%%%%%%%

\section*{Course Objectives}
You will master many aspects of Linux/Unix, and acquaint yourself with others. You will learn:
\begin{multicols}{2}
\begin{enumerate}
\item bash
\item vim
\item awk
\item sed
\item network programming
\item file system management
\item version control
\item webservers
\item Linux account management
\item run scheduled tasks
\item ( a little ) C.
\item kernel driver development
\item containerization/virtualization.
\item ...and more...
\end{enumerate}
\end{multicols}

% Fifth Section %%%%%%%%%%%%%%%%%%%%%%%%%%%%%%%%%%%%%%%%%%%

\section*{Course Structure}

\subsection*{Class Structure}

This material will likely be new to you, and it's very interesting. There will be weekly lectures and assignments as well as two exams. Every week we will cover a new aspect of Linux / Unix / Programming / System Administration and there will be a related assignment due the following lecture. In the event that you miss a class there will be related reading materials posted online so you can still complete your assignment.

\subsubsection*{Lecture}

Each week in class we will log on to our virtual machines and explore a specific aspect of programming in Linux.

\subsection*{Assessments}

Weekly programming assignments based on the materials covered in class. 2 exams bringing together various aspects of topics taught. No late assignments. Lowest weekly assignment grade will be dropped. Multiple choice midterm and final exams.

\subsection*{Grading Policy}
Grading is as follows:

\begin{itemize}
	\item \underline{\textbf{20\%}} Midterm exam.
	\item \underline{\textbf{20\%}} Final exam.
	\item \underline{\textbf{60\%}} Average of weekly assignments. 
\end{itemize}

For example, if you get an 80 on exam 1, an 85 on exam 2, and ( to simplify assume there were only two weekly assignments ) your weekly assignment scores are [ 90, 95 ], your final score will be 20pts * 0.80 + 20pts * 0.85 + 60pts * ( 0.90 + 0.95) / 2 = 88.5 pts, and will be translated to the appropriate letter grade following NJCU guidelines.

\newpage
\section*{Schedule and weekly learning goals}

The schedule is tentative and subject to change. The learning goals below should be viewed as the key concepts you should grasp after each week, and also as a study guide before each exam.
\newpage

\subsection*{Date: 01/24 - Introduction to Linux, Bash, Vim }
\begin{itemize}
\item Setup and connect to computer
\item How to use github + git ( brief intro )
\item Learn basic BASH commands:
\begin{multicols}{2}
\begin{enumerate}
    \item ls (-l -la -a)
    \item hidden files
    \item cd
    \item cp
    \item mv
    \item whoami
    \item which
    \item mkdir
    \item touch
    \item rm
    \item touch
    \item cat (assignment1)
    \item echo (assignment1)
    \item man
    \item wc (assignment1)
    \item md5sum (assignment1)
    \item df
    \item du
    \item uniq
    \item sort
    \item find
    \item xargs
    \item more
    \item less
    \item grep ( no regex yet )
    \item unzip (assignment1)
    \item tar
    \item cut (assignment1)
    \item if (assignment1)
    \item arrays
    \item Return values and \$? (assignment1)
    \item tree
    \item etc.
\end{enumerate}
\end{multicols}
\item wildcards
\item bash variables ( export vs without export and subshells )
\item bash eq, not equal
\item vim ( modes, p, shif+p, :wq, :x, i, a, dd, arrows, <esc>, skipping around. set nu/nonu. ctrl+v. search+replace s///g s///gc. :sort. :sort n. u for undo, ctrl+r. ctrl+a to increment. split vs vsplit. +/- for next line. 5w = 5 words. 5l = 5 letters. )
\item pipes | (assignment1)
\item telnet starwars
\item csh, zsh, sh, dash, etc.
\end{itemize}

\subsection*{Date: 01/24 - Introduction to Linux, Bash, Vim }
\begin{itemize}
	\item Review last week + a Little More Bash
	\item Review git usage
	\item File Descriptors ( stdin, stdout, stderr )
	\item Processes
\end{itemize}

\subsection*{Date: 01/24 - Introduction to Linux, Bash, Vim }
\begin{itemize}
	\item cron
	\item more bash
	\item about git
\end{itemize}


\subsection*{Date: 01/24 - Introduction to Linux, Bash, Vim }
\begin{itemize}
	\item more bash
	\item ssh + sftp
\end{itemize}

\subsection*{Week 5}
\begin{itemize}
	\item File System Hierarchy
	\item Mounting and Partitioning
	\item Move Home directory
	\item PATH
	\item Swap space
\end{itemize}

\subsection*{Week 6}
\begin{itemize}
	\item FileDescriptors
	\item First Programs
	\item Signal handlers
	\item setuid + setgid
\end{itemize}

\subsection*{Week 7}
\begin{itemize}
	\item Installing software
	\item symlinks
	\item ASCII
	\item more bash
\end{itemize}

\subsection*{Week 8}
\begin{itemize}
	\item cURL
	\item Install cURL from source
	\item HTTP verbs
	\item REST
\end{itemize}

\subsection*{Week 9}
\begin{itemize}
	\item playing with curl, setting up servers
	\item set up a simple webserver with PythonSimpleHTTP
	\item set up a simple webserver with Flask
	\item make requests to our servers.
	\item set up website with Apache
\end{itemize}

\subsection*{Week 10}
\begin{itemize}
	\item Awk
\end{itemize}

\subsection*{Week 11}
\begin{itemize}
	\item UTF-8
	\item sed
\end{itemize}

\subsection*{Week 12}
\begin{itemize}
	\item init
	\item services
	\item systemd vs sysvinit vs upstart
\end{itemize}

\subsection*{Week 13}
\begin{itemize}
	\item gpg
	\item encryption
\end{itemize}

\subsection*{Week 14}
\begin{itemize}
	\item Set up a docker container
	\item Set up a personal git server
\end{itemize}
\end{document}
